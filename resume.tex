%&tex
% http://kb.mit.edu/confluence/pages/viewpage.action?pageId=3907092

\documentclass[10pt]{article}
\usepackage{geometry}
\geometry{
 % a4paper,
 papersize={8.5in,11in},
 % total={170mm,257mm},
 left=20mm,
 right=20mm,
 top=3mm,
 bottom=10mm
 }
\usepackage{titling}
\setlength{\droptitle}{-1.5em}
\usepackage{graphicx}
\newcommand\sbullet[1][.5]{\mathbin{\vcenter{\hbox{\scalebox{#1}{$\bullet$}}}}}
\usepackage{hyperref}
\hypersetup{
     colorlinks=true,
     linkcolor=blue,
     filecolor=blue,
     citecolor = black,
     urlcolor=blue,
}
\usepackage{url}
\usepackage{multicol}
\usepackage[dvipsnames]{xcolor}
\definecolor{lighterB}{RGB}{0,40,180}
\definecolor{lighterG}{RGB}{60,60,70}

\newcommand{\CC}{C\nolinebreak\hspace{-.05em}\raisebox{.4ex}{\tiny\bf +}\nolinebreak\hspace{-.10em}\raisebox{.4ex}{\tiny\bf +}}

\newcommand\myemail{
  \href{mailto:piratach@gmail.com} 
  {\underline{piratach@gmail.com}}
}

\title{\bfseries\Huge Piratach Yoovidhya}
\renewcommand\maketitlehookb{\vspace{-5ex}}
\author{
  \small \myemail
   $\sbullet[.75]$
   % (+1) 412-773-2027
   (+1) 412-636-8372
   % $\sbullet[.75]$
   % +668-0868-9988
   $\sbullet[.75]$
   \href{https://github.com/Piratach}{\underline{Github}}
   $\sbullet[.75]$
  \href{https://www.linkedin.com/in/piratach-yoovidhya/}
  {\underline{Linkedin}}
}
\date{}
\setlength\parindent{0pt}
\pagestyle{empty}

\begin{document}
  \maketitle
  \thispagestyle{empty}
  \vspace*{-1.0cm}
  \hrule
  \vspace*{-0.15cm}

  \section*{\Large \textcolor{lighterB}{Education}}
  \vspace*{-0.3cm}

  \textbf{\large \textcolor{Black}{Carnegie Mellon University $\sbullet$ Pittsburgh, PA}}

  \vspace{0.05cm}

  \textcolor{Black}{\textbf{M.S. in Computer Science, \underline{\href{https://github.com/Piratach/thesis-document/blob/main/CMU-CS-24-143.pdf}{Research Thesis}}} (GPA: 4.0)} \hfill \textit{August 2024}

  \vspace{0.05cm}

\textcolor{Black}{\textbf{B.S. in Computer Science, Concentration in Computer Systems}} \hfill \textit{May 2022}

  \vspace{0.2cm}
  % \begin{itemize}
    % \itemsep-0.4em
    % \item GPA: 3.37, Dean's List
    % \item Dean's List: Fall 2018
    % \item Member of the Game Creation Society, The Atlas Project

  \vspace*{-0.5cm}

  \subsection*{Selected \textcolor{Black}{Coursework:}}

    \vspace*{-0.55cm}
    \begin{multicols}{2}
       15-740, Computer Architecture \\
       15-410, Operating Systems \\
       \columnbreak
       15-451, Algorithm Design and Analysis \\
       17-715, Hardware Security \\
       15-852, Parallel and Concurrent Algorithms \\
       15-744, Graduate Computer Networks \\
    \end{multicols}

  % \vspace*{-1.2cm}
  \vspace*{-1.0cm}
  \section*{\Large \textcolor{lighterB} {Work Experience}}
  % \vspace*{-0.23cm}
  % \hrule depth 0.2mm \relax
  % \vspace{0.2cm}

\textbf{\large Apple Inc. $\sbullet$} {\large Graphics Modeling Engineer $\sbullet$ \textit{Cupertino, CA}} {\hfill \textit{\textbf{(Aug 2024 - Present)}}}

  \vspace*{-0.2cm}
  \begin{itemize}
    \itemsep-0.4em
    \item \textcolor{lighterG}{Performance modeling for GPU memory systems.}
    % \item \textcolor{lighterG}{Worked with Professor Robert Murphy in implementing \textit{Bioactive}, a program that is used to assist in research through active learning and model construction}
    % \item \textcolor{lighterG}{Fixed database issues that prevented several campaigns from working as intended}
    % \item \textcolor{lighterG}{Implemented a continuous modeler based on linear regression and modularized the code for other files}
  \end{itemize}


\textbf{\large Google LLC $\sbullet$} {\large Software Engineer $\sbullet$ \textit{Sunnyvale, CA}} {\hfill \textit{\textbf{(Aug 2022 - July 2023)}}}

  \vspace*{-0.2cm}
  \begin{itemize}
    \itemsep-0.4em
\item \textcolor{lighterG}{Worked within Google Cloud Storage, designing and implementing a load generator used to generate prod-representative traffic.}
    \item \textcolor{lighterG}{This is used to ensure changes are robust, and will not cause any issues when rolled out to production.}
  \end{itemize}

\textbf{\large KBTG $\sbullet$} {\large Data Science Intern $\sbullet$ \textit{Bangkok, Thailand}} {\hfill \textit{\textbf{(Jun 2019 - Aug 2019)}}}

  \vspace*{-0.2cm}
  \begin{itemize}
    \itemsep-0.4em
    \item \textcolor{lighterG}{Worked in the data science team to develop a feature that evaluated the price of a car (for collateral) from a photo to be used in K-Plus, Thailand's \#1 mobile banking app}
    \item \textcolor{lighterG}{Successfully developed a license plate and vehicle image recognition model using Keras, and connected it to a pipeline that would function as a part of the vehicle price evaluation program}
  \end{itemize}

  % \vspace*{-0.55cm}

    \section*{{\Large \textcolor{lighterB}{Research Experience}} {\normalsize \textit{\textcolor{lighterB}{(at Carnegie Mellon University)}}}}
  % \vspace*{-0.23cm}
  % \hrule depth 0.2mm \relax
  % \vspace{0.2cm}

  \textbf{\large Simulating Cache Coherence for Cache-Attached Accelerators} {\hfill \textit{\textbf{(Aug 2023 - Aug 2024)}}}
  \vspace*{-0.17cm}
  \begin{itemize}
    \itemsep0em
    \item \textcolor{lighterG}{Builds on top of existing work on \href{https://ieeexplore.ieee.org/document/10106564}{\underline{Kobold}}, through a \textbf{model} in the \textbf{gem5} simulator. (\textbf{C++}, \textbf{Python})}
    \item \textcolor{lighterG}{Designed and implemented a novel memory access predictor,
        novel replacement policies, and micro-benchmarks.}
    \item \textcolor{lighterG}{In collaboration with Professor Nathan Beckmann and PhD student Jennifer Brana.}
  \end{itemize}

  \textbf{\large Microarchitectural Simulation of Polymorphic Cache Hierarchy} {\hfill \textit{\textbf{(Jan 2022 - May 2022)}}}
  \vspace*{-0.17cm}
  \begin{itemize}
    \itemsep0em
    \item \textcolor{lighterG}{Simulated microarchitecture through a dataflow architecture on a CGRA in a \textbf{Zsim-based} simulator in \textbf{C++}.}
    \item \textcolor{lighterG}{Allows for fine-grained instruction-level parallelism and flexibility over routing of inputs/outputs.}
    \item \textcolor{lighterG}{In collaboration with Professor Nathan Beckmann and PhD students Brian Schwedock and Nikhil Agarwal.}
  \end{itemize}

  \textbf{\large \href{https://dl.acm.org/doi/10.1145/3470496.3527379}{\underline{täkō}}: a polymorphic cache hierarchy} {\hfill \textit{\textbf{(Nov 2020 - May 2022)}}} 
  \vspace*{-0.17cm}
  \begin{itemize}
    \itemsep0em
    \item \textcolor{lighterG}{Designed and implemented applications that show a significant speedup compared to the baseline cache.}
    \item \textcolor{lighterG}{Identified potential problem in the system, where the callbacks would pollute the core's L2 with unused data. This is later addressed in \href{https://ieeexplore.ieee.org/document/10106564}{\underline{Kobold}}.}
    \item \textcolor{lighterG}{Best paper nominee at ISCA'22.}
    \item \textcolor{lighterG}{Collaborated with Professor Nathan Beckmann and PhD student Brian Schwedock.}
  \end{itemize}

  % \vspace{0.2cm}
  \vspace{0.5cm}
  \textbf{\large{Notable Skills}}: C, C++, Python, CUDA, Modelling, Memory Systems, Accelerators, gem5, CPU Architecture, \\ GPU Architecture, Low-power Computing, Cache Design

\end{document}
